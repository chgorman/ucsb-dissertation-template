%%%%%%%%%%%%%%%%%%%%%%%%%%%%%%%%%%%%%%%%%%%%%%%%%%%%%%%%%%%%%%%%%%%%%%%%
%%% Preamble: New Commands

% Shorthand for Chebyshev-Vandermonde
\newcommand{\CV}[0]{C\=/V} % This requires the extdash package

\newcommand{\norm}[1]{\left| \left| #1 \right| \right|}		
\newcommand{\pd}[2]{\frac{\partial #1}{\partial #2}}
\newcommand{\angles}[1]{\left\langle #1 \right\rangle}
\newcommand{\abs}[1]{\left| #1 \right|}
\newcommand{\opernorm}[1]{\left|\left|\left| #1 \right| \right| \right|}

\newcommand{\parens}[1]{\left( #1 \right)}
\newcommand{\brackets}[1]{\left[ #1 \right]}	
\newcommand{\braces}[1]{\left\{ #1 \right\}}
\newcommand{\lprline}[1]{\left. #1 \right|}

\DeclareMathOperator{\sgn}{sgn}
\DeclareMathOperator{\sign}{sign}
\DeclareMathOperator{\tr}{tr}
\DeclareMathOperator{\rank}{rank}
\DeclareMathOperator{\Span}{Span}
\DeclareMathOperator{\diag}{diag}
\DeclareMathOperator{\supp}{supp}
\DeclareMathOperator{\Ker}{Ker}

\newcommand{\del}[0]{\delta}
\newcommand{\eps}[0]{\varepsilon}
\newcommand{\ego}[0]{\varepsilon > 0}% Let epsilon be greater than 0...
\newcommand{\vphi}[0]{\varphi}
\newcommand{\N}[0]{\mathbb{N}}
\newcommand{\R}[0]{\mathbb{R}}
\newcommand{\Q}[0]{\mathbb{Q}}
\newcommand{\Z}[0]{\mathbb{Z}}
\newcommand{\C}[0]{\mathbb{C}}
\newcommand{\E}[0]{\mathbb{E}}
\newcommand{\V}[0]{\mathbb{V}}
\renewcommand{\P}[0]{\mathbb{P}} % Overwrites Paragraph symbol
\newcommand{\RP}[0]{\mathbb{RP}}
\newcommand{\pdl}[3]{\left.\frac{\partial #1}{\partial #2} \right|_{#3}}
\newcommand{\dfl}[2]{\left. d #1 \right|_{#2}}
\newcommand{\csu}[3]{\Gamma_{#1 #2}^{#3}}
\newcommand{\csl}[3]{\Gamma_{#1 #2 #3}}
\newcommand{\Ric}[0]{\text{Ric}\,}
\newcommand{\id}[0]{\text{id}}

\newcommand{\epsa}[0]{\varepsilon_{\text{abs}}}
\newcommand{\epsr}[0]{\varepsilon_{\text{rel}}}

% New definition of square root:
% it renames \sqrt as \oldsqrt
\let\oldsqrt\sqrt
% it defines the new \sqrt in terms of the old one
\def\sqrt{\mathpalette\DHLhksqrt}
\def\DHLhksqrt#1#2{%
\setbox0=\hbox{$#1\oldsqrt{#2\,}$}\dimen0=\ht0
\advance\dimen0-0.2\ht0
\setbox2=\hbox{\vrule height\ht0 depth -\dimen0}%
{\box0\lower0.4pt\box2}}



% Increase number of allowable matrix columns
\setcounter{MaxMatrixCols}{20}


%\newtheoremstyle{mythm}% name of the style to be used
%  {parskip}% measure of space to leave above the theorem. E.g.: 3pt
%  %{spaceabove}% measure of space to leave above the theorem. E.g.: 3pt
%  {parskip}% measure of space to leave below the theorem. E.g.: 3pt
%  %{spacebelow}% measure of space to leave below the theorem. E.g.: 3pt
%  {}% name of font to use in the body of the theorem
%  %{bodyfont}% name of font to use in the body of the theorem
%  {\quad}% measure of space to indent
%  %{indent}% measure of space to indent
%  {}% name of head font
%  %{headfont}% name of head font
%  {}% punctuation between head and body
%  %{headpunctuation}% punctuation between head and body
%  {}% space after theorem head; " " = normal interword space
%  %{headspace}% space after theorem head; " " = normal interword space
%  {}
%  %{\thmname{#1}\thmnumber{ #2}:\thmnote{ #3}}
%  %{headspec}% Manually specify head

\newtheoremstyle{mythm}% name
  {12pt}%Space above
  %{3pt}%Space above
  {12pt}%Space below
  {\normalfont}%Body font
  {0pt}%Indent amount
  {\bf}% Theorem head font
  %{\itshape}% Theorem head font
  {}%Punctuation after theorem head
  %{.}%Punctuation after theorem head
  {\newline}%Space after theorem head 2
  {}%Theorem head spec (can be left empty, meaning ‘normal’)

\theoremstyle{mythm}
\newtheorem{thm}{Theorem}[chapter] % number by chapter; can replace with section
\newtheorem{cor}[thm]{Corollary}
\newtheorem{lem}[thm]{Lemma}
\newtheorem{prop}[thm]{Proposition}
\newtheorem{ax}{Axiom}
\newtheorem{problem}[thm]{Problem}
%
\theoremstyle{definition}
\newtheorem{defn}{Definition}[chapter] % number by chapter
%
\theoremstyle{remark}
\newtheorem{rem}{Remark}[chapter]
\newtheorem*{notation}{Notation}
%
\newtheorem{hw}[thm]{Homework}



